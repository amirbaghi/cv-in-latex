%-------------------------------------------------------------------------------
%	SECTION TITLE
%-------------------------------------------------------------------------------
\cvsection{Projects}


%-------------------------------------------------------------------------------
%	CONTENT
%-------------------------------------------------------------------------------
\begin{cventries}

%---------------------------------------------------------

        \cventry
        {EA SEED and Uppsala University - Master's Thesis} % Organisation
        {Applying Multi-Agent Reinforcement Learning as Game-AI in Football-like Environments} % Project
        {\textbf{Stockholm, Sweden}} % Location
        {\textbf{Spring and Summer 2024}} % Date(s)
        {
            A state-of-the-art multi-agent reinforcement learning method was researched, implemented, and improved upon, using reward shaping and architectural changes, as the game-AI for the game of football. Implemented mainly in PyTorch and Ray, among other tools. Under supervision of Linus Gisslén, Alessandro Sestini, and Jens Sjölund. 
        }

        
        \cventry
        {Uppsala University - Independent Research Work and Project in Computer Systems Course} % Organisation
        {Neural Compression and Novel View Synthesis of High-Resolution Micro-CT Imaging} % Project
        {\textbf{Uppsala, Sweden}} % Location
        {\textbf{Winter 2023}} % Date(s)
        {
            Implicit neural representations (INRs), like CuNeRF and hash-based methods, were used to compress, reconstruct, and render high-resolution 3D micro-CT scans. Implemented using PyTorch and Optuna. Under supervision of Jens Sjölund and partly Stefanos Kaxiras in the course.
        }
        
       \cventry
        {Uppsala University - Reinforcement Learning Course} % Organisation
        {Deep-Q Network for Pong} % Project
        {\textbf{Uppsala, Sweden}} % Location
        {\textbf{Spring 2023}} % Date(s)
        {
            A Deep Q-Network (DQN) was implemented, based on the Nature paper by Mnih et al. (2015), to learn and play the game of Pong (and most Atari games). The network was tested on "Breakout!" successfully as well. Implemented using PyTorch.
        }

       % \ifbool{FullVersion}{
       % \cventry
       %  {Uppsala University - Global Software Product Development Course} % Organisation
       %  {Prototype For Semi-Autonomous Agricultural Harvesting Robots} % Project
       %  {\textbf{Uppsala, Sweden}} % Location
       %  {\textbf{Spring 2023}} % Date(s)
       %  {
       %      Collaborated internationally with students from HUST (Hanoi) and UU (Uppsala) to develop a semi-autonomous agricultural harvesting robot prototype using Lego EV3 Robots, emphasizing teamwork and project management. Mainly programmed the movement/navigation of robots.
       %  }
       % }

     % \ifbool{FullVersion}{
       % \cventry
       %  {Uppsala University - Statistical Machine Learning Course} % Organisation
       %  {Various Classifiers and a DL Model For Lead Role Prediction In Films} % Project
       %  {\textbf{Uppsala, Sweden}} % Location
       %  {\textbf{Spring 2023}} % Date(s)
       %  {
       %      Five types of classifiers and a neural network were developed with advanced hyper-tuning for prediction of the lead actor's gender based on a script's attributes. Implemented using PyTorch, RayTune, and scikit-learn. Got a \emph{golden star} in the course.
       %  }
     % }{}

    % \ifbool{FullVersion}{  
    % \cventry
    % {Uppsala University - Computer Graphics Course} % Organisation
    % {Raytracing, Environment, and Shadow Mapping For a Model Viewer} % Project
    % {\textbf{Uppsala, Sweden}} % Location
    % {\textbf{Spring 2023}} % Date(s)
    % {
    %     Enhanced an existing GLTF model viewer as the final course project by adding raytracing, environment, and shadow mapping. Developed in C++ with OpenGL and GLSL, building upon earlier assignments that established the renderer's foundations.
    % }
    % }{}
  
  % \ifbool{FullVersion}{
  % \cventry
  %   {Uppsala University - Human-Computer Interaction Course} % Organisation
  %   {A Game Listing App UI} % Project
  %   {\textbf{Uppsala, Sweden}} % Location
  %   {\textbf{Fall 2022}} % Date(s)
  %   {
  %       Based on human-computer interaction principles, a user interface was developed for a fictional video game listing app. Developed in Adobe XD.
  %   }
  % }{}

  % \ifbool{FullVersion}{
  % \cventry
  %   {Uppsala University - Combinatorial Optimisation Course} % Organisation
  %   {Added Constraints, MiniCP Solver} % Project
  %   {\textbf{Uppsala, Sweden}} % Location
  %   {\textbf{Fall 2022}} % Date(s)
  %   {
  %       New constraints (e.g. AllDifferent and Disjunctive) were developed for the MiniCP constraint solver in Java.
  %   }
  % }{}

  % \ifbool{FullVersion}{  
  % \cventry
  %   {Uppsala University - Combinatorial Optimisation Course} % Organisation
  %   {Carcassonne Map Generator} % Project
  %   {\textbf{Uppsala, Sweden}} % Location
  %   {\textbf{Fall 2022}} % Date(s)
  %   {
  %       A \emph{proper} map generator for the game of Carcassonne was implemented in a constraint modeling language, namely MiniZinc. Specific constraints and rules were to be followed in the generation.
  %   }
    % }{}
    
  \cventry
    {Uppsala University - Functional Programming I Course} % Organisation
    {Advanced Reversi AI} % Project
    {\textbf{Uppsala, Sweden}} % Location
    {\textbf{Fall 2022}} % Date(s)
    {
        The game of Reversi (Othello) and an advanced opponent AI were implemented in Haskell. Ranked \emph{3rd} in a tournament against other AIs.
    }

  % \ifbool{FullVersion}{
  % \cventry
  %   {Shiraz University - Digital System Design Course} % Organisation
  %   {The CORDIC Machine} % Project
  %   {\textbf{Shiraz, Iran}} % Location
  %   {\textbf{Fall 2021}} % Date(s)
  %   {
  %     The circular-rotation mode of the CORDIC machine was to be implemented on FPGA. Simulated in Python and written in Verilog.
  %   }
  %   % }{}

    % \ifbool{FullVersion}{
      \cventry
        {Shiraz University - Systems Analysis and Design Course} % Organisation
        {Aphrodite, A Covid-19 Detection Application} % Project
        {\textbf{Shiraz, Iran}} % Location
        {\textbf{Fall 2021}} % Date(s)
        {
            Developed a Dockerized COVID-19 detection application for Shiraz hospitals using a pre-trained model (COVIDNet-CT). Utilized Python, Qt C++, Prometheus, Grafana, Node Exporter, and MySQL.
        }
    % }{}
% 
    % \ifbool{FullVersion}{
  % \cventry
  %   {Shiraz University - Databases Design Principles Course} % Organisation
  %   {Alethia, Mock Linkedin Website} % Project
  %   {\textbf{Shiraz, Iran}} % Location
  %   {\textbf{Spring 2021}} % Date(s)
  %   {
  %     As the course project, a website simulating the features and looks of LinkedIn was developed. Vue.js, Gin (Golang), MySQL, Docker, gRPC, cron, Redis, and Nginx were utilized. All services were dockerized and an effective container orchestration was designed. \emph{Ranked 2nd in the course}.
  %   }
    % }{}

  % \ifbool{FullVersion}{
   \cventry
    {Shiraz University - Computer Games Design and Computer Graphics I Courses} % Organisation
    {Game Engine Development Mini-Projects} % Project
    {\textbf{Shiraz, Iran}} % Location
    {\textbf{Fall 2020 - Spring 2021}} % Date(s)
    {
        Implemented game engine components in C++ with OpenGL/GLFW/GLUT, incorporating core game development and graphics concepts such as skeletal animation, game physics, rendering engines, design patterns, and metric analysis.
    } 
  % }{}


  % \ifbool{FullVersion}{
  % \cventry
  %   {Shiraz University - Operating Systems Course} % Organisation
  %   {Simple Unix Bash} % Project
  %   {\textbf{Shiraz, Iran}} % Location
  %   {\textbf{Fall 2020}} % Date(s)
  %   {
  %     As the course project, a simple Unix-like bash was written in C++ to simulate the main functionalities of bash (e.g. executing Unix commands).
  %   }
  % }{}

 % \ifbool{FullVersion}{
  % \cventry
  %   {Shiraz University - Computer Graphics I Course} % Organisation
  %   {3d Eight Queens} % Project
  %   {\textbf{Shiraz, Iran}} % Location
  %   {\textbf{Fall 2020}} % Date(s)
  %   {
  %     As the course project, a program simulating the eight queens problem in 3D graphics was written in C++ using only OpenGL. Advanced lighting and surface materials were implemented and used to improve the graphics.
  %   }
 % }

  % \ifbool{FullVersion}{
  % \cventry
  %   {Shiraz University - Technical Presentation Course \emph{(Volunteer Project)}} % Organisation
  %   {Presentation/Video Sharing Website} % Project
  %   {\textbf{Shiraz, Iran}} % Location
  %   {\textbf{Fall 2020}} % Date(s)
  %   {
  %     Done voluntarily for the Technical Presentation course. Based on the suggestion of the instructor of the course, Dr. Koorush Ziarati, we developed a website where students can share presentations and give feedback on other submissions, including a management panel for the instructor and an online video-streaming option with live comments. Developed using Django and Vue.js.
  %   }
    % }{}

  % \ifbool{FullVersion}{
    % \cventry
    % {I.D.E.A. Footsteps Contest} % Organisation
    % {Modo, Bus Transportation App} % Project
    % {\textbf{Shiraz, Iran}} % Location
    % {\textbf{Summer 2020}} % Date(s)
    % {
    %   For the I.D.E.A. Footsteps Contest, we were tasked with implementing a tracking system for the university’s bus lines, including a mobile app to provide users with info on the scheduling. Developed with Flutter. \emph{Ranked 1st in the contest}.
    % }
  % }{}

  % \ifbool{FullVersion}{  
    % \cventry
    %     {Shiraz University - Computer Architecture and Computer Architecture Lab Courses} % Organisation
    %     {MIPS-Based CPU} % Project
    %     {\textbf{Shiraz, Iran}} % Location
    %     {\textbf{Spring 2020 - Fall 2020}} % Date(s)
    %     {
    %       A pipelined CPU based on MIPS was implemented in Python (\emph{for Computer Arch. Course, ranked 1st}) and in Verilog (for Computer Arch. Lab Course).
    %     }
    % }{}

% \ifbool{FullVersion}{
  % \cventry
  %   {Shiraz University - Data Structures and Algorithms I Course} % Organisation
  %   {Steiner Tree Finder} % Project
  %   {\textbf{Shiraz, Iran}} % Location
  %   {\textbf{Spring 2020}} % Date(s)
  %   {
  %     As the course project, an algorithm to find an acceptable Steiner Tree for a graph, in O(N) time complexity, was designed and implemented in Python.
  %   }
% }{}

  % \ifbool{FullVersion}{
  % \cventry
  %   {Shiraz University - Advanced Programming Course} % Organisation
  %   {Atelier, Workshop Management Website} % Project
  %   {\textbf{Shiraz, Iran}} % Location
  %   {\textbf{Fall 2019}} % Date(s)
  %   {
  %     As the course project, our team implemented a fully functional website, written in Java, to carry out a workshop management system with various features. \emph{The project was ranked 1st}.
  %   }
  % }{}

  % \ifbool{FullVersion}{
  % \cventry
  %   {Shiraz University - Principles of Programming Course} % Organisation
  %   {Teatro, Game Console In C} % Project
  %   {\textbf{Shiraz, Iran}} % Location
  %   {\textbf{Spring 2019}} % Date(s)
  %   {
  %     As the course project, our team implemented a game console to be able to render certain 2D games on PC, written in C. \emph{Ranked 2nd in the course}.
  %   }
   % }{} 

  % \ifbool{FullVersion}{
  % \cventry
  %   {Shiraz University - Fundamentals of Computer and Programming Course} % Organisation
  %   {Blockchain-Based Transaction Management System} % Project
  %   {\textbf{Shiraz, Iran}} % Location
  %   {\textbf{Fall 2018}} % Date(s)
  %   {
  %     As the course project, a system to manage transactions in an unreal currency, based mainly on blockchain technology, was implemented in Python.
  %   }
  % }{}

%---------------------------------------------------------
\end{cventries}
